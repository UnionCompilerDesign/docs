%%%%%%%%%%%%%%%%%%%%%%%%%%%%%%%%%%%%%%%%%
% fphw Assignment
% LaTeX Template
% Version 1.0 (27/04/2019)
%
% This template originates from:
% https://www.LaTeXTemplates.com
%
% Authors:
% Class by Felipe Portales-Oliva (f.portales.oliva@gmail.com) with template 
% content and modifications by Vel (vel@LaTeXTemplates.com)
%
% Template (this file) License:
% CC BY-NC-SA 3.0 (http://creativecommons.org/licenses/by-nc-sa/3.0/)
%
%%%%%%%%%%%%%%%%%%%%%%%%%%%%%%%%%%%%%%%%%

%----------------------------------------------------------------------------------------
%	PACKAGES AND OTHER DOCUMENT CONFIGURATIONS
%----------------------------------------------------------------------------------------

\documentclass[
	12pt, % Default font size, values between 10pt-12pt are allowed
	%letterpaper, % Uncomment for US letter paper size
	%spanish, % Uncomment for Spanish
]{fphw}

% Template-specific packages
\usepackage[utf8]{inputenc} % Required for inputting international characters
\usepackage[T1]{fontenc} % Output font encoding for international characters
\usepackage{mathpazo} % Use the Palatino font

\usepackage{graphicx} % Required for including images

\usepackage{booktabs} % Required for better horizontal rules in tables

\usepackage{listings, listings-rust} % Required for insertion of code
\usepackage{llvm}

\usepackage{enumerate} % To modify the enumerate environment

%----------------------------------------------------------------------------------------
%	ASSIGNMENT INFORMATION
%----------------------------------------------------------------------------------------

\title{Assignment \#X: Generating Code} % Assignment title


\date{DUE DATE} % Due date

\institute{Union College} % Institute or school name

\class{CSC-375 Compiler Design} % Course or class name

\professor{Aaron Cass} % Professor or teacher in charge of the assignment

%----------------------------------------------------------------------------------------

\begin{document}

\maketitle % Output the assignment title, created automatically using the information in the custom commands above

%----------------------------------------------------------------------------------------
%	ASSIGNMENT CONTENT
%----------------------------------------------------------------------------------------

\section*{Objectives}

\begin{problem}
	\begin{itemize}
        \item Complete the last part of code compilation, generating LLVM assembly from the output of your semantic analyzer
        \item Pass all tests given in \textit{ir\_tests.rs}
        \item Work in groups of 3-4
        
	\end{itemize}
\end{problem}

%------------------------------------------------

\subsection*{An explanation of code generation with LLVM IR:}
At long last, this is it! This is the step of compilation that you've been waiting for, actually generating code runnable by a computer! All of the steps before this have been preparing you for this point, and it's finally time to take the ASTs that you have semantically analyzed and turn that into code.\\
\\
If you're here from CSC-270, you might be a little nervous that you have to write assembly out by hand, but this won't be what we do here. Instead, we'll be using LLVM IR (Intermediate Representation) code. LLVM IR is a higher level assembly-like language which can itself be turned into proper assembly for your processor so that you can write the code here and have it work on many CPUs, much like the same Java code can be compiled and run on many CPUs.  
\pagebreak
%----------------------------------------------------------------------------------------

\section*{LLVM IR Example}
    \lstinputlisting[
		caption=code for a while loop in C, % Caption above the listing
		label=lst:Cllexample, % Label for referencing this listing
		language=c, % Use Rust functions/syntax highlighting
		frame=single, % Frame around the code listing
		showstringspaces=false, % Don't put marks in string spaces
		numbers=left, % Line numbers on left
		numberstyle=\tiny, % Line numbers styling
	]{file.c} % file name to list

	\lstinputlisting[
		caption=equivalent code for a while loop in LLVM IR, % Caption above the listing
		label=lst:llexample, % Label for referencing this listing
		language=llvm, % Use Rust functions/syntax highlighting
		frame=single, % Frame around the code listing
		showstringspaces=false, % Don't put marks in string spaces
		numbers=left, % Line numbers on left
		numberstyle=\tiny, % Line numbers styling
	]{output_while_loop.ll} % file name to list

This step of code compilation will bridge the gap between code in C and code in LLVM IR, completing the transformation from a program string to usable code. This may seem like a daunting task, but you've been provided with a set of tools that will make writing this out much easier.\\
\\
In \textit{core.rs}, the IRGenerator struct is what will be used to generate IR code from a given AST, and it uses another struct called IRManager. IRManager is the link between your code and generating LLVM code. For example, if you need to create an integer, the \textit{create\_integer()} function of the resource pool will handle this for you.\\
\\
At the heart of this is the tag system. When you call \textit{create\_integer()} for example, what it will give back is a ValueTag. Think of tags as unique names that are linked to the object you've created, so when you call \textit{create\_integer()} it creates the value, stores it, and then gives you a tag that represents it, sort of like a ticket you can use to get your dry cleaning.\\
\\
Another key idea is the basic block. A basic block is simply a section of code that you can write instructions into. The code above for the while loop has four basic blocks: entry, while\_cond, while\_body, and while\_end. Each are separate basic blocks that code is written inside.


\section*{code walk: core.rs}
This is the file where the main router function that handles generating IR is. This time, a fair amount of this will be given to you completed, as the focus of this part is learning how to generate IR. You must design the IR generation loop (remember the parser!), write the functions that the router points to, and generate IR for these cases appropriately. 

\section*{code walk: primitive.rs}
This file contains the functions that generate IR for data types and literal values like integers.

\section*{code walk: statement.rs}
This file contains the functions that generate IR for statements like returns and breaks, as well as variable assignment. Anything which doesn't evaluate to a value goes here.

\section*{code walk: block.rs}
This file contains the functions that generate IR for complex code blocks like if statements and loops, anything which generates basic blocks.

\section*{code walk: store.rs}
This file contains an implementation of a temporary store to help you keep track of variable tags. This is given, you don't have to do anything here. There are functions in IRGenerator which bind to this structure for you.


\pagebreak

\section*{Follow these guidelines to build the IR Generator struct effectively}

\begin{problem}
    \begin{enumerate}
        \item Do not change any of the function signatures, i.e. leave all parameters, parameter types, and return types as you found them.
        \item Follow all instructions given in the comments where they exist, and take any potential hints into consideration when making your design.
        \item Keep track of tags! Each tag is your ticket to something you created in resource pools.
        \item Remember to create a basic block every time you need a new section of code.
        \item Remember that at this stage you've already done semantic analysis, you may assume the AST you're recursing over is always correctly formatted at this point!
        \item \textcolor{red}{IMPORTANT:} To simulate a real production environment, this compiler will eventually contain multithreading, and so we had to implement thread safety for the resource pools. What this means for you is that every time you use a function of resource pools, you must first get the resource pools, try to lock the resource pools (expecting a panic error is ok here because otherwise we couldn't continue anyway), work with the resource pools through the lock object, and drop the lock when you're done. A diagram of this process in code is shown below and you can see this for yourself in action in the \textit{generate\_literal\_ir()} function.

        
    \end{enumerate}
\end{problem}

\lstinputlisting[
		caption=Process for working with the resource pools after locking, % Caption above the listing
		label=lst:lockexample, % Label for referencing this listing
		language=Rust, % Use Rust functions/syntax highlighting
		frame=single, % Frame around the code listing
		showstringspaces=false, % Don't put marks in string spaces
        breaklines=true,
		numbers=left, % Line numbers on left
		numberstyle=\tiny, % Line numbers styling
	]{sample.rs} % file name to list
\pagebreak


\end{document}
